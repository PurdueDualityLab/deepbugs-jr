\section{Conclusions}
Although there are many tools to help software engineers identify and mitigate bugs, such as fuzzers, linters, and unit test frameworks, deep learning provides another avenue to develop tools which can support the software development process. The original DeepBugs study addressed the problem of name-based bug detection through deep learning. The study was able to produce detectors with high accuracy. Following the procedure outlined in the study, our team was able to replicate results with similar accuracy and precision. In the future, our team plans to share our results with the software engineering community by submitting the results to a suitable venue (such as the Journal of Open Source Software). To provide further insights into DeepBugs' capabilities, we currently consider to further test DeepBugs beyond the 150k JavaScript Dataset, either with a different corpus of JavaScript code or with different programming languages entirely (this will require modification of a few DeepBugs elements). Implementing DeepBugs for a variety of languages can assist software engineers in the detection of these name-based bugs and increase the quality of software across the board.


% In the original study, the target language was JavaScript to test the viability of name-based bug detection with a dynamically-typed language. In the future, similar technologies could be 

% reported differences and stuff