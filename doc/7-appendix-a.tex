\section{Appendix A: Ethics Assessment}
\label{sec:appendix_a}
% Dario
\subsection{Dario}
Just as software engineers must submit their code for review to their peers, so too must the authors of research papers submit their work to the academic community. In a similar fashion, the authors offer up their work for the scrutiny of their peers to find validation. This process has ethical implications that go beyond the opinions of the reviewers. In this case, our attempt to replicate the Deepbugs study found similar results to that specified in the original study. As such, the time our group spent this semester serves to fortify the claims made in the original study and provide additional, independent proof to the validity of this work.

% Abhimanyu

\subsection{Abhimanyu}
Conducting replication and reproduction studies of previously conducted research is one of the hallmarks of good science. Reproducibility provides scientists with evidence that research results are objective and reliable and not due to bias or chance 
\cite{rooney_how_2016}. Replication and reproduction cumulatively help in addressing ethical issues such as data falsification. In this project, the team worked on implementing the work done in DeepBugs. This helped in verifying and validating results put forth by the author of DeepBugs. Fabrication of results by the authors could lead to major ethical issues.

It is noteworthy that while the original work made use of JavaScript extensively, our implementation is in Python. It is entirely possible to see different results due to difference in programming languages being used as the code written would see variation. As engineers following ACM Code of Ethics, one must be responsible and avoid public harm. The results put forth by DeepBugs along with the result validation done gives credibility to the original author's work. An automated bug detection tool like DeepBugs would certainly prove to be useful to the engineering society.

An ethically ideal software engineer would be an individual who is honest, responsible, and courageous. These aspects are crucial because these would impact the choice a software engineer makes in difficult ethical circumstances. Moreover, these traits may not be accounted by the code of ethics that most engineers are expected to follow, they essentially come from within due to the moral values one has always been bounded with. Being honest with others and towards your work is extremely important. An honest software engineer takes the responsibility and comes out to be accountable for the work done. 

Throughout our research study, we followed the ACM Code of Ethics and ensured that the replication study carried out eliminates any concerns on reproducibility. This idea ensures public welfare and ensures that no software without a strong fundamental foundation and improper testing is not proposed for public use. 

% Caleb

\subsection{Caleb}
A replication study lends credibility to an existing piece of research. In our case, we validated the results of the DeepBugs paper. By reimplementing the original authors' work entirely in Python (they used JavaScript as well), we provided more ways for the public to use the paper, making it even more accessible. Although this does support aspects of the ACM/IEEE Code of Ethics -- we help ensure that the authors stay honest with the public -- there is another ethical consideration: is the original work something that deserves the credibility we give it via replication?

I believe the original DeepBugs paper is a good contribution to science and engineering. By automating new types of bug detection, the paper provides another tool for engineers to use in their ongoing war against defects in software. Further, I find it difficult to imagine any lasting ramifications of the paper that would violate any point of the Code of Ethics. Therefore, I find this paper to be worthy of elevating via replication.

% Young
\subsection{Young} Replication and reproduction studies are essential to keep the research community healthy and reliable, by primarily allowing the other researchers to adopt, verify and possibly extend the implementation with confidence. As noted earlier, the community recognizes this as a ``reproducibility crisis.''
An ethically ideal person is courageous without manifesting the common virtue or excellence to the maximum extent possible, beyond what’s ordinary. Harris Jr. describes this as a ``virtue portrait'' in his article. \cite{harris_good_2008}
With the technical and non-technical excellence, one can attempt to sketch a virtue portrait of the ethically ideal software engineer. It is our belief that all software engineers should attempt to picture virtue portrait on themselves while practicing software engineering especially by contemplating and possessing an utmost interest in the social outcome of one's action. We believe that our practice of replication helps mitigate the reproducibility crisis and has positive ethical social outcomes by addressing the following clauses of ACM-IEEE Code of Ethics: \textit{1.03. Approve software only if they have a well-founded belief that it is safe, meets specifications, passes appropriate tests, and does not diminish the quality of life, diminish privacy or harm the environment. The ultimate effect of the work should be to the public good.} and \textit{3.10. Ensure adequate testing, debugging, and review of software and related documents on which they work.} 

% Jordan

\subsection{Jordan}
Replication and reproduction studies are part of healthy scientific progress. Repeatability of observation is a core part of the scientific method. The scientific method has been a staple formula for both the natural and social sciences since its inception. Reproduction studies are also important in the formal sciences. In the formal sciences, it is more so called publishing an alternate proof, instead of a reproduction study. In the formal sciences, it is important to do this for sake of understanding the result from a different perspective. It is also the case that results in formal sciences, if complex enough, may disagree. \cite{lipton_social_1979} Replication is often done ``automatically'' as a result of re-creating a proof when one reads a paper. But having multiple alternate published proofs adds confidence in the correctness of a result.

\iffalse
% ethical word fodder below: MIT licensed
DeepBugs falls into the category of being both an empirical scientific result and a formal one. Although computer science is categorized as part of the formal sciences, Artificial Intelligence methods which involve learning, are potentially different from this categorization. This is due to the fact that learning methods can use empirical data. And so, for a machine learning algorithm such as DeepBugs, behavior can be different under a different set of data. 
% Also note the attempt at replication of 150 DNN's here
As a formal result, DeepBugs uses a particular implementation with JavaScript and Python. The result could be different if implemented using different code. That is not say there are any issues with Python and JavaScript in particular, or programming skills of the researchers from the previous paper. The choice is not due to recent statistical analysis on programming languages versus problem domains. \cite{berger_impact_2019,ray_large_nodate}
The aim instead is to implement a solution in a different way, because it is different. And Python and TensorFlow in particular are chosen because our group's skills mostly overlap on these tools. 


There are also ethical concerns for robustness of this research for use in engineering. By the ACM code of ethics, we must avoid harm. \cite{noauthor_code_2016} The result from DeepBugs, and results of similar impact, should be used in a way that minimizes harm. Replication of results increases confidence in those results, before they are used in products that effect people. Replication of results also follows the ethical standard that engineers should strive to be honest and trustworthy. Ignoring due diligence to replicate results for sake of personal gain is at the cost of people that rely on these results.


%% covered several times in above sections
% A successfully replicated study will confirm the result from DeepBugs. Extensions to DeepBugs can be made with higher confidence of correctness. A negative result will draw attention to any potential weaknesses in the methodology from either this paper or the original, encouraging refinement on the result from DeepBugs. 

%import Code_of_Ethics_for_Engineers 

\fi

